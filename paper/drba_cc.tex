% this file is drba_tp.tex
% title page
\section{Conclusion}
\label{sec:concl}

In this thesis a software was introduced to compute Gröbner fans for linear codes. Firstly, the mathematical background and concepts were discussed with the purpose to develop the software, with the hope that it might be useful for researches and other thesis that only needs the degree compatible Gröbner fan.\\
\\~
In my knowledge, no other software provides this useful feature to compute the degree compatible Gröbner fan. Computing the degree compatible can be dramatically faster than computing the whole Gröbner fan.\\
\\~
The given data structures and the well written algorithms from TiGERS \cite{tigers} made it easy to extend the software with a lot of new features in terms of linear codes and the degree compatible Gröbner fan. The computational experience in section \ref{subsec:compexp} shows that CIDGEL can be way faster than Gfan for computing the whole Gröbner fan. Computing the degree compatible Gröbner fan with CIDGEL leads to an additional speedup.