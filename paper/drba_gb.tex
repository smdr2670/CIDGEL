%this file is drba_gb.tex
\section{Polynomial and commutative Algebra}

In this chapter a mathematical basis is systematically approached to give the reader an understanding to Groebner Bases and obtaining by the Flipping-Algorithm which is needed later.\\
In the first section monomials are revisited.
The second section explains how monomials can be mathematically ordered.
After that Ideals are defined over polynomial rings and a summary on Groebner bases and Groebner fans for ideals is presented.

\subsection{Monomials}
\newtheorem{env_definition}{Definition}[section]

First of all, the basic components of a polynomial ring...

\begin{env_definition}[Monomial] 

A monomial m is a product of variables over a finite field $\mathbb{K}$, denoted by $ \mathbb{K} \left[X_{1},X_{2},\cdots X_{n}\right]  $ of the form $X_{1}^{u_{1}}X_{2}^{u_{2}}\cdots X_{n}^{u_{n}}$, where $u_{i}, 1 < i < n $ and $u \in \mathbb{N}_{0} $

The total \textbf{degree} of a monomial is $deg(m) = \sum_{i=1}^n u_i $ 
\end{env_definition}

\begin{env_definition}[Polynomial]

A polynomial f is a finite linear combination with coefficients $c_{u} \in \mathbb{K}$ multiplied with monomials.

$ f = \sum_{u} c_{u}X^{u}$\\

If $c_{u}\neq0$ then $c_{u}x_{u}$ is a term of $f$


\end{env_definition}


\subsection{Monomial Order}

It is necessary to arrange the terms of a polynomial in order to compare every pair of polynomials. That is important for instance dividing polynomials in the finite field
$ \mathbb{K} \left[X_{1},X_{2},\cdots X_{n}\right]  $

\begin{env_definition}[Term Ordering] 



\end{env_definition}


\subsection{Ideals}



\newpage