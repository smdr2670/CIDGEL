\section{Introduction}

~\\
\subsection{Motivation}
The Gröbner fan of a polynomial ideal is a polyhedral complex consisting of cones in $\mathbb{R}_{+}^{n}$.
Many applications of Gröbner bases rely on the computation of the Gröbner fan.
Gröbner bases have the nice property that they solve the Ideal membership problem and can be useful to solve polynomial equations.
Furthermore, the Gröbner fan can be used for a necessary condition for the code equivalence problem. \\
In many applications, the complete Gröbner fan is not needed, but only the degree compatible. 
Computing this certain part of the Gröbner fan could be dramatically faster than computing the whole fan.

~\\

\subsection{Tasks}
The purpose of this bachelor thesis is to implement an efficient software, which enumerates the degree compatible Gröbner fan of a linear code. But first, the concepts and algorithms have to be researched. The mathematical background is discussed and then the implementation and its results are explained. 

\newpage

\subsection{Structure}
Chapter \ref{sec:background} deals with the mathematical background. The first subsections explain polynomials, term ordering, ideals and the Ideal Membership Problem which is necessary for the Gröbner bases and Gröbner fans. \\
Chapter \ref{sec:enumerate} bases on chapter \ref{sec:background}. The algorithms to enumerate the Gröbner fan of a linear code are introduced which is the main task to implement in the software.\\
Chapter \ref{sec:software} presents the basic data structures of the implementation. Furthermore this section shows a tutorial how to use the software and some computational experience.\\
Chapter \ref{sec:future} gives an overview of the future possible extensions and improvements.\\
In Chapter \ref{sec:concl} conclusions about the results are drawn.
\newpage