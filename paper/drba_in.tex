\section{Introduction}

\subsection{Motivation}
Code equivalent problem and so on...

\subsection{Tasks}
The purpose of this bachelor thesis is to implement an efficient software, which enumerates the degree compatible Gröbner fan of a linear code, but first the concepts and algorithms have to be researched. The mathematical background is discussed first and then the implementation is and its results are explained. 


\subsection{Structure}
Chapter 2 deals with the mathematical background. The first subsections explains polynomials, term ordering, ideals and the ideal-membership problem which is sufficient for the Gröbner bases and Gröbner fans. Based on these, the algorithms to enumerate the Gröbner fan are introduced which is the main task to implement in the software.\\
Chapter 3 presents the basic data structures of the implementation. Furthermore this section shows a tutorial how to use the software and some computational experience.\\
Chapter 4 gives an overview of the future possible extensions and improvements.\\
In Chapter 5 conclusions about the results are drawn.
\newpage