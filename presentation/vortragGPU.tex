\documentclass{beamer}
\usepackage{beamerthemelined} 
\usepackage{amsmath}
\usepackage{mathtools}
\usepackage{listings}
\usepackage{graphicx}
\usepackage{epstopdf}

\usepackage{amsmath,amsfonts,amssymb}

\usepackage[ngerman]{babel}

%\usepackage[UTF8]{inputenc}

\setbeamertemplate{footline}[frame number]

\lstdefinestyle{customc}{
  belowcaptionskip=1\baselineskip,
  breaklines=true,
  frame=L,
  xleftmargin=\parindent,
  language=C++,
  showstringspaces=false,
  basicstyle=\footnotesize\ttfamily,
  keywordstyle=\bfseries\color{green!40!black},
  commentstyle=\itshape\color{purple!40!black},
  identifierstyle=\color{blue},
  stringstyle=\color{orange},
  tabsize=2,
}

%\usepackage{ stmaryrd }

%\usetheme{Boadilla} % Pretty neat, soft color.
%\usetheme{default}
%\usetheme{Warsaw}
%\usetheme{Bergen} % This template has nagivation on the left
%\usetheme{Frankfurt} % Similar to the default 
%with an extra region at the top.
%\usecolortheme{seahorse} % Simple and clean template
%\usetheme{Darmstadt} % not so good
% Uncomment the following line if you want %
% page numbers and using Warsaw theme%
 %\setbeamertemplate{footline}[page number]
%\setbeamercovered{transparent}
\setbeamercovered{invisible}
% To remove the navigation symbols from 
% the bottom of slides%
\setbeamertemplate{navigation symbols}{} 
%

\usepackage{graphicx}
%\usepackage{bm}         % For typesetting bold path (not \mathbold)
%\logo{\includegraphics[height=0.6cm]{yourlogo.eps}}
%
\title[Gröbner-Fächer für lineare Codes]{Gröbner-Fächer für lineare Codes}
\author{Daniel Rembold}
\institute[TUHH]
{
Technische Universit{\"a}t Hamburg Harburg \\
\medskip
{\emph{daniel.rembold@tuhh.de}}
}
\date{\today}
% \today will show current date. 
% Alternatively, you can specify a date.
%


\graphicspath{%
    {converted_graphics/}% inserted by PCTeX
    {/}% inserted by PCTeX
}
\begin{document}
% \renewcommand{\name}{Abbildung}
%
\begin{frame}
\titlepage
\end{frame}
%

\begin{frame}
\frametitle{Inhaltsverzeichnis}

\begin{enumerate}
			\item Einleitung \\

	      	\item Mathematische Grundlagen \\
	      	
	      	\item Aufzählen von Gröbner-Fächern  \\ 
	      		
	      	\item Ergebnisse  \\ 
	  
	      	\item Fazit \\

		    \item Vorführung
\end{enumerate}

\end{frame}

%%%%%%%%%%%%%%%%%%%%%%%%%%%%%%%%%% BEGIN INTRODUCTION %%%%%%%%%%%%%%%%%%%%%%%%%%%%%%%%%
\section{Einleitung}

\begin{frame}
\frametitle{Motivation}

\begin{itemize}
\item
Gröbnerbasen 

\item



\end{itemize}






\end{frame}

%%%%%%%%%%%%%%%%%%%%%%%%%%%%%%%%%%%%%%%%%%%%%%%%%%%%

\section{Mathematische Grundlagen}



%%%%%%%%%%%%%%%%%%%%%%%%%%%%
\begin{frame}[fragile]
\frametitle{Monome }

\begin{block}{Monom}
\begin{itemize}
\item Produkt von Variablen über ein endliches Feld $ \mathbb{K} \left[X_{1},X_{2},\dots, X_{n}\right]  $ 
\item Schreibweise $m= X_{1}^{u_{1}}X_{2}^{u_{2}}\cdots X_{n}^{u_{n}}$ und $u_i \in \mathbb{N}_{0}  $
\end{itemize}

\end{block}

Grad eines Monoms:  $deg(m) = \sum_{i=1}^n u_i $. 




\end{frame}

% % % % % % % % % % % % % % % % % % % % % % % % % % % % %
\begin{frame}[fragile]
\frametitle{Termordnung}
% --  Kommentar 
% --  Zu erst wird die Definition von Termordnung erklärt , dann werden die verschieden arten
\begin{block}{Termordnung $>$}
\begin{itemize}
\item Relation $>$ zu der Menge von allen Monomen in $ \mathbb{K} \left[X_{1},X_{2},\dots, X_{n}\right]  $ 
\end{itemize}



\end{block}

\begin{block}{Termordnung }
\begin{itemize}
\item Lexikographische Ordnung $>_{lex}$
\item grad $>_{grlex}$
\item Ordnung mit Gewichtsvektor $c = (c_1, \ldots , c_n ) \in \mathbb{R}^{n}$
\end{itemize}



\end{block}


\end{frame}
% % % % % % % % % % % % % % % % % % % % % % % % % % % % % 
\begin{frame}[fragile]
\frametitle{Leitterm}

\begin{block}{Leitterm \textsc{LT}$(f)$ }
\begin{itemize}
\item Polynom $p \in  \mathbb{K} \left[X_{1},X_{2},\dots, X_{n}\right] $  besitzt Term höchster Ordnung in Bezug auf $>$
\end{itemize}

\end{block}


\begin{block}{Beispiel}
Sei $f = x^{2}+3xyz+y^{3} $
\begin{itemize}

\item lex-Order : $f = \underline{x^{2}} + 3xyz + y^{3}  $
\item grlex-Order :  $f = \underline{3xyz} + y^{3} + x^{2}  $
\item $\left(1,2,1\right)$ :  $f = \underline{y^{3}} + 3xyz + x^{2}  $
\end{itemize}

\end{block}


\end{frame}
% % % % % % % % % % % % % % % % % % % % % % % % % % % % %
\begin{frame}[fragile]
\frametitle{Ideale}

\begin{block}{Ideal }
\begin{itemize}
\item Kollektion von Polynomen $f_{1}, \dots , f_{s}$ : \\
\[ \langle f_{1}, \dots , f_{s} \rangle = \left\lbrace  \sum_{i=1}^s h_{i}f_{i} \mid h_{1}, \dots , h_{s} \in \mathbb{K}\left[X_{1}, \dots, X_{n}\right] \right\rbrace. \] 
 
\end{itemize}

\end{block}

\begin{block}{Beispiel}
Sei $ I= \langle f_{1},f_{2} \rangle = \langle x^{2}+y, x+y+1 \rangle $ und $f=x^{2}y+x^{2}+y^{2}+xy+x$ \\
Dann gilt $f= y \cdot f_{1} + x \cdot f_{2},~f\in I$.

\end{block}


\end{frame}

\begin{frame}
\frametitle{Divisionsalgorithmus}




	
\end{frame}
%%%%%%%%%%%%%%%%%%%%%%%%%%%


\begin{frame}[fragile]
\frametitle{NVIDIA-Template}

\begin{itemize}
\item
Zerlegung der Matrix in Bl\"ocken
\item
Verwendung von lokalem Speicher 
\end{itemize}





\end{frame}

%  % %  % % % % % % % % % %  % % %

\begin{frame}[fragile]
\frametitle{Gröbner Basis}


\end{frame}



% % % % % % % % % % % % % % % % % % % % %
\begin{frame}[fragile]
\frametitle{Gröbner Fächer}





\end{frame}


\begin{frame}[fragile]
\frametitle{Torische Ideale}



\end{frame}
% -------------------------------------------------------- % % %
\section{Aufzählen von Gröbner-Fächern}


\begin{frame}
\frametitle{BLABLA}

\end{frame}

\begin{frame}
\frametitle{Performancemessung auf ATI und NVIDIA}


\end{frame}


\section{Fazit}

\begin{frame}
\frametitle{Vergleich}


	
\end{frame}


\begin{frame}
\frametitle{M\"ogliche Verbesserungen}


 
\end{frame}













%%%%%%%%%%%%%%%%%%%%%%%%%%%%%%%%%
%%%%%%%%%%%%%%%%%%%%%%%%%%%%%%%%%
%%%%%%%%%%%%%%%%%%%%%%%%%%%%%%%%%

\begin{frame}
\frametitle{Quellen}


\footnotesize{
\begin{thebibliography}{99}
 \bibitem[Label1, 2010]{key1} University of Bristol
 \newblock Optimizing OpenCL performance
 \newblock \url{http://www.cs.bris.ac.uk/home/simonm/workshops/OpenCL_lecture3.pdf} 
\end{thebibliography}
}



\footnotesize{
\begin{thebibliography}{99}
 \bibitem[Label1, 2010]{key1}NVIDIA 
 \newblock OpenCL SDK Code Samples
 \newblock \url{https://developer.nvidia.com/opencl} 
\end{thebibliography}
}


\footnotesize{
\begin{thebibliography}{99}
 \bibitem[Label1, 2010]{key1} Vasily Volkov
(UC Berkeley , September 22, 2010)
 \newblock Better Performance at Lower Occupancy 
 \newblock \url{http://www.cs.berkeley.edu/~volkov/volkov10-GTC.pdf} 
\end{thebibliography}
}





\end{frame}
 
 
 
\begin{frame}
\centerline{Vielen Dank f{\"u}r eure Aufmerksamkeit!}
\end{frame}
% End of slides
\end{document}
