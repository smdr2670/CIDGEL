\documentclass{beamer}
\usepackage[UTF8]{inputenc}

\usepackage{beamerthemelined} 
\usepackage{amsmath}
\usepackage{mathtools}
\usepackage{listings}
\usepackage{graphicx}
\usepackage{epstopdf}
\usepackage{amsmath,amsfonts,amssymb}




\usepackage{algorithm}% http://ctan.org/pkg/algorithms
\usepackage{algpseudocode}% http://ctan.org/pkg/algorithmicx

\newcommand{\Input}{\item[\algorithmicinput]}
\newcommand{\algorithmicinput}{\textbf{Input:}}

\newcommand{\Output}{\item[\algorithmicoutput]}
\newcommand{\algorithmicoutput}{\textbf{Output:}}



\usepackage{tikz}
\usetikzlibrary{decorations.markings}
% \usepackage{courier}
% \usepackage[default,osfigures,scale=0.95]{opensans}

\usepackage{subcaption}
\usepackage{caption}
\usetikzlibrary{backgrounds}

%\usepackage[ngerman]{babel}

%% Für PC Windows
%\usepackage[ansinew]{inputenc}


\setbeamertemplate{footline}[frame number]

% Syntax für C
\lstdefinestyle{customc}{
  belowcaptionskip=1\baselineskip,
  breaklines=true,
  frame=L,
  xleftmargin=\parindent,
  language=C++,
  showstringspaces=false,
  basicstyle=\footnotesize\ttfamily,
  keywordstyle=\bfseries\color{green!40!black},
  commentstyle=\itshape\color{purple!40!black},
  identifierstyle=\color{blue},
  stringstyle=\color{orange},
  tabsize=2,
}

%\usepackage{ stmaryrd }

\usetheme{Madrid}
%\usetheme{Boadilla} % Pretty neat, soft color.
%\usetheme{default}
%\usetheme{Warsaw}
%\usetheme{Bergen} % This template has nagivation on the left
%\usetheme{Frankfurt} % Similar to the default 
%with an extra region at the top.
%\usecolortheme{seahorse} % Simple and clean template
%\usetheme{Darmstadt} % not so good
% Uncomment the following line if you want %
% page numbers and using Warsaw theme%
 %\setbeamertemplate{footline}[page number]
%\setbeamercovered{transparent}
\setbeamercovered{invisible}
% To remove the navigation symbols from 
% the bottom of slides%
\setbeamertemplate{navigation symbols}{} 
%

\usepackage{graphicx}
%\usepackage{bm}         % For typesetting bold path (not \mathbold)
%\logo{\includegraphics[height=0.6cm]{yourlogo.eps}}
%
\title[Gröbner-Fächer für lineare Codes]{Gröbner-Fächer für lineare Codes}
\author{Daniel Rembold}
\institute[TUHH]
{
Technische Universit{\"a}t Hamburg Harburg \\
\medskip
{\emph{daniel.rembold@tuhh.de}}
}
\date{\today}
% \today will show current date. 
% Alternatively, you can specify a date.
%


\graphicspath{%
    {converted_graphics/}% inserted by PCTeX
    {/}% inserted by PCTeX
}
\begin{document}
% \renewcommand{\name}{Abbildung}
%
\begin{frame}
\titlepage
\end{frame}
%

\begin{frame}
\frametitle{Inhaltsverzeichnis}

\begin{enumerate}
			\item Einleitung \\

	      	\item Mathematische Grundlagen \\
	      	
	      	\item Aufzählen von Gröbner-Fächern  \\ 
	      		
	      	\item Ergebnisse  \\ 
	  
	      	\item Fazit \\

		    \item Vorführung
\end{enumerate}

\end{frame}

%%%%%%%%%%%%%%%%%%%%%%%%%%%%%%%%%% BEGIN INTRODUCTION %%%%%%%%%%%%%%%%%%%%%%%%%%%%%%%%%
\section{Einleitung}

\begin{frame}
\frametitle{Motivation}

\begin{itemize}
\item
Gröbnerbasen 

\item



\end{itemize}






\end{frame}

%%%%%%%%%%%%%%%%%%%%%%%%%%%%%%%%%%%%%%%%%%%%%%%%%%%%

\section{Mathematische Grundlagen}



%%%%%%%%%%%%%%%%%%%%%%%%%%%%
\begin{frame}[fragile]
\frametitle{Monome }

\begin{block}{Monom}
\begin{itemize}
\item Produkt von Variablen über ein endliches Feld $ \mathbb{K} \left[X_{1},X_{2},\dots, X_{n}\right]  $ 
\item Schreibweise $m= X_{1}^{u_{1}}X_{2}^{u_{2}}\cdots X_{n}^{u_{n}}$ und $u_i \in \mathbb{N}_{0}  $
\end{itemize}

\end{block}

Grad eines Monoms:  $deg(m) = \sum_{i=1}^n u_i $. 




\end{frame}

% % % % % % % % % % % % % % % % % % % % % % % % % % % % %
\begin{frame}[fragile]
\frametitle{Termordnung}
% --  Kommentar 
% --  Zu erst wird die Definition von Termordnung erklärt , dann werden die verschieden arten
\begin{block}{Termordnung $>$}
\begin{itemize}
\item Relation $>$ zu der Menge von allen Monomen in $ \mathbb{K} \left[X_{1},X_{2},\dots, X_{n}\right]  $ 
\end{itemize}



\end{block}

\begin{block}{Termordnung }
\begin{itemize}
\item Lexikographische Ordnung $>_{lex}$
\item grad $>_{grlex}$
\item Ordnung mit Gewichtsvektor $c = (c_1, \ldots , c_n ) \in \mathbb{R}^{n}_{+}$
\end{itemize}



\end{block}


\end{frame}
% % % % % % % % % % % % % % % % % % % % % % % % % % % % % 
\begin{frame}[fragile]
\frametitle{Leitterm}

\begin{block}{Leitterm \textsc{LT}$(f)$ }
\begin{itemize}
\item Polynom $p \in  \mathbb{K} \left[X_{1},X_{2},\dots, X_{n}\right] $  besitzt Term höchster Ordnung in Bezug auf $>$
\end{itemize}

\end{block}


\begin{block}{Beispiel}
Sei $f = x^{2}+3xyz+y^{3} $
\begin{itemize}

\item lex-Order : $f = \underline{x^{2}} + 3xyz + y^{3}  $
\item grlex-Order :  $f = \underline{3xyz} + y^{3} + x^{2}  $
\item $\left(1,2,1\right)$ :  $f = \underline{y^{3}} + 3xyz + x^{2}  $
\end{itemize}

\end{block}


\end{frame}
% % % % % % % % % % % % % % % % % % % % % % % % % % % % %
\begin{frame}[fragile]
\frametitle{Ideale}

\begin{block}{Ideal }
\begin{itemize}
\item Kollektion von Polynomen $f_{1}, \dots , f_{s}$ : \\
\[ \langle f_{1}, \dots , f_{s} \rangle = \left\lbrace  \sum_{i=1}^s h_{i}f_{i} \mid h_{1}, \dots , h_{s} \in \mathbb{K}\left[X_{1}, \dots, X_{n}\right] \right\rbrace. \] 
 
\end{itemize}

\end{block}

\begin{block}{Beispiel}
Sei $ I= \langle f_{1},f_{2} \rangle = \langle x^{2}+y, x+y+1 \rangle $ und $f=x^{2}y+x^{2}+y^{2}+xy+x$ \\
Dann gilt $f= y \cdot f_{1} + x \cdot f_{2},~f\in I$.

\end{block}


\end{frame}

\begin{frame}
\frametitle{Divisionsalgorithmus (1)}



\begin{itemize}
\item Notwendig zum Lösen des Idealzugehörigkeitsproblems


\end{itemize}

\begin{block}{Ziel des Algorithmus}
Polynom g durch Ideal $ I = \langle f_{1}, \dots , f_{s} \rangle  $ teilen, so dass 
$ g = a_{1}f_{1} + \ldots + a_{s}f_{s} + r ,~~~ a_{i},g,I,r \in \mathbb{K}\left[X_{1}, \dots, X_{n}\right]  $
\end{block}

~\\

Sei Polynom $p$ und Ideal $I$  

\begin{itemize}
\item Wenn p \% I = 0, dann gilt $ p \in I$


\end{itemize}



	
\end{frame}
%%%%%%%%%%%%%%%%%%%%%%%%%%%


\begin{frame}[fragile]
\frametitle{Divisionsalgorithmus (2)}




\begin{algorithm}[H]
\caption{Divisionsalgorithmus (Header) }

\begin{algorithmic}[1]

\Require Basis $I = \langle f_{1}, \dots, f_{m}\rangle$ of nonzero polynomials \textbf{and} ~~~~~~~~~~~~~~~~~~~~ a fixed termorder \textsc{LT}  
\Ensure $r=0$ \textbf{or} none of the terms in $r$ are divisible by $ LT_{\leq}\left( f_{1}\right) , \dots , LT_{\leq} \left( f_{m}\right) $

%\EndProcedure
\end{algorithmic}
\end{algorithm}


% Hier Beispiel mit verdrehten Idealen zeigen
\begin{itemize}
\item Reihenfolge der Polynome in $I$ beeinflusst Ergebnis
\item $r \neq 0$ möglich, obwohl $p \in I$ 
\end{itemize}


\end{frame}



%  % %  % % % % % % % % % %  % % %

\begin{frame}[fragile]
\frametitle{Gröbnerbasis (1)}
%Kommentar : B


\begin{block}{Gröbnerbasis  }
Sei Termordnung $>$ und Ideal $I$, dann hat Gröbnerbasis $G = \{f_{1}, \ldots, f_{m} \}$ (in Bezug auf $>$) von $I$ die Eigenschaft: 
\begin{itemize}
\item Von jedem Polynom $p \in I$ ist $\textsc{LT}_{>}(p)$ teilbar durch $\textsc{LT}_{>}(f_{i})$  
\end{itemize} 
\end{block}

~\\
\begin{itemize}
\item Divisionsrest eindeutig bestimmt und unabhängig von Reihenfolge
~\\
\item Gröberbasis aus jedem Ideal mit Hilfe des Buchberger-Algorithmus und einer festen Termordnung 
\end{itemize}



\end{frame}



% % % % % % % % % % % % % % % % % % % % %

\begin{frame}[fragile]
\frametitle{Gröbnerbasis (2)}
%Kommentar 

\begin{itemize}
\item Gröbernbasen sind nicht eindeutig

\item Reduzierte Gröbnerbasen jedoch eindeutig für Ideale
\end{itemize}

\begin{block}{Reduzierte Gröbnerbasis}
Alle Leitterme von Gröbnerbasis G in Bezug auf Termordnung $>$ monisch und relativ prim zueinander 
\end{block}




\end{frame}













% % % % % % % % % % % % % % % % % % % % % % % % % %
\begin{frame}[fragile]
\frametitle{Gröbner-Fächer}
%Kommentar: Text der Bachelorarbeit
\begin{itemize}
\item Unendlich viele Termordnungen, endlich viele reduzierte Gröbnerbasen

\end{itemize}

\begin{block}{Gröbner-Fächer}
\begin{itemize}
\item Vielflächiges Komplex welches Kegel im $\mathbb{R}^{n}_{+}$ enthält
\item Flächen werden durch Ebenenungleichungen der Polynome bestimmt 
\end{itemize}

\end{block}

\end{frame}
% % % % % % % % % % % % % % % % % % % % % % % % % %

\begin{frame}[fragile]
\frametitle{Beispiel Gröbner-Fächer (1)}
 Sei 
 \begin{enumerate}
 \item $I = \langle x^{2}-z,~y-x  \rangle \in \mathbb{K}\left[x,y,z\right]  $
 \item $G_{>_{lex}} = \{ \underline{y^{2}}-z,~\underline{x}-y  \} $ 
 \item $\textbf{w}= (a,b,c) \in \mathbb{R}^{3}_{+} $
\end{enumerate}
 
~\\
\begin{itemize}
\item $w \in G_{>_{lex}}$ genau dann, wenn 
~\\
~\\
\begin{enumerate}
\item $\left(0,2,0\right) \cdot \left(a,b,c\right) \geq \left(0,0,1\right) \cdot \left(a,b,c\right) \vee 2b \geq c  $
~\\
~\\
\item $\left(1,0,0\right) \cdot \left(a,b,c\right) \geq \left(0,1,0\right) \cdot \left(a,b,c\right) \vee a \geq b $ 
\end{enumerate}
\end{itemize}

\end{frame}


% % % % % % % % % % % % % % % % % % % % % % % % % %
\begin{frame}[fragile]
\frametitle{Beispiel Gröbner-Fächer (2)}

\begin{center}


\begin{figure}
\caption{Gröbner-Kegel für $G_{>_{lex}}$}

\begin{tikzpicture}[join=round,scale=0.8]
    \tikzstyle{conefill} = [fill=blue!20,fill opacity=0.8]
    \tikzstyle{ann} = [fill=white,font=\footnotesize,inner sep=1pt]
    \tikzstyle{ghostfill} = [fill=white]
    \tikzstyle{ghostdraw} = [draw=black!50]
    
    \draw[arrows=->,line width=.4pt](0,0,0)--(0,0,5); %Z_achse
    \draw[arrows=->,line width=.4pt](0,0,0)--(0,5,0); %Y-ACHSE
    \draw[arrows=->,line width=.4pt](0,0,0)--(5,0,0); %X-ACHSE
    %\draw[arrows=<-,line width=.4pt](.42,-.767)--(4,-2);
    
    \path (5,0,0) node[below] {$b$} (0,0,5) node[above] {$a$} (0,5,0) node[left] {$c$};
    
% letzte Koordinate ist A!!!
% zweite Koordinate ist C!!!
% dritte Koordinate ist B!!!
  
\filldraw[conefill](0,0,0)--(0,0,4)--(1,2,1)--cycle;
\filldraw[conefill](1,2,1)--(0,0,4)--(2,0,2)--cycle;
%\filldraw[conefill](1,2,1)--(0,0,0)--(2,0,2)--cycle;

\draw [opacity=0.2] (0,0,0) -- (2,0,2) ;
 
   
\end{tikzpicture}

\end{figure}

\end{center}



\end{frame}

% % % % % % % % % % % % % % % % % % % % % % % % % % 
\begin{frame}[fragile]
\frametitle{Beispiel Gröbner-Fächer (3)}

\begin{center}


\begin{figure}
\caption{Kompletter Gröbner-Fächer}

\begin{tikzpicture}[join=round,scale=0.8]
    \tikzstyle{conefill} = [fill=blue!20,fill opacity=0.8]
    \tikzstyle{ann} = [fill=white,font=\footnotesize,inner sep=1pt]
    \tikzstyle{ghostfill} = [fill=white]
    \tikzstyle{ghostdraw} = [draw=black!50]
    
    \draw[arrows=->,line width=.4pt](0,0,0)--(0,0,5); %Z_achse
    \draw[arrows=->,line width=.4pt](0,0,0)--(0,5,0); %Y-ACHSE
    \draw[arrows=->,line width=.4pt](0,0,0)--(5,0,0); %X-ACHSE
    %\draw[arrows=<-,line width=.4pt](.42,-.767)--(4,-2);
    
    \path (5,0,0) node[below] {$b$} (0,0,5) node[above] {$a$} (0,5,0) node[left] {$c$};
   
% erste Koordinate ist B!!! 
% zweite Koordinate ist C!!!   
% dritte Koordinate ist A!!!


% äußere Flächen!

\filldraw[conefill](0,0,0)--(0,4,0)--(0,0,4)--cycle;
\filldraw[conefill](0,0,0)--(4,0,0)--(0,0,4)--cycle;
\filldraw[conefill](0,0,0)--(4,0,0)--(0,4,0)--cycle;

% untenfläche
\filldraw[conefill](0,0,0)--(0,0,4)--(2,0,2)--cycle;
\filldraw[conefill](0,0,0)--(4,0,0)--(2,0,2)--cycle;

%
\filldraw[conefill](0,0,4)--(2,0,2)--(1,2,1)--cycle;
\filldraw[conefill](4,0,0)--(2,0,2)--(1,2,1)--cycle;

\filldraw[conefill](4,0,0)--(1,2,1)--(0,4,0)--cycle;
\filldraw[conefill](0,0,4)--(1,2,1)--(0,4,0)--cycle;

% zur Übersichtlichkeit die gerade im "inneren" des Cones
\draw [opacity=0.2] (0,0,0) -- (1,2,1) ;
\end{tikzpicture}


\end{figure}

\end{center}


\end{frame}



% % % % % % % % % % % % % % % % % % % % % % % % % %

\begin{frame}[fragile]
\frametitle{Torische Ideale}

\begin{itemize}
\item Obermenge von Code-Idealen
\item Besteht nur aus Binomen
\end{itemize}

\begin{block}{Torisches Ideal}
Gegeben seien $A =\left[a_{1},\dots, a_{n}  \right] \in \mathbb{Z}^{d \times n } $ und $u \in \mathbb{Z}^{n}$ zerlegbar in $u^{+}$ und $u^{-}$, dann ist das torische Ideal $I_{A}$ definiert durch \\
~\\
\centering
$ \textbf{I}_{A} = \langle \textbf{x}^{u^{+}} - \textbf{x}^{u^{-}} \mid u \in ker \left(  A \right) \rangle . $ 

\end{block}


\end{frame}
% -------------------------------------------------------- % % %
\section{Aufzählen von Gröbner-Fächern}


\begin{frame}
\frametitle{BLABLA}

\end{frame}

\begin{frame}
\frametitle{Performancemessung auf ATI und NVIDIA}


\end{frame}


\section{Fazit}

\begin{frame}
\frametitle{Vergleich}


	
\end{frame}


\begin{frame}
\frametitle{M\"ogliche Verbesserungen}


 
\end{frame}













%%%%%%%%%%%%%%%%%%%%%%%%%%%%%%%%%
%%%%%%%%%%%%%%%%%%%%%%%%%%%%%%%%%
%%%%%%%%%%%%%%%%%%%%%%%%%%%%%%%%%

\begin{frame}
\frametitle{Quellen}


\footnotesize{
\begin{thebibliography}{99}
 \bibitem[Label1, 2010]{key1} University of Bristol
 \newblock Optimizing OpenCL performance
 \newblock \url{http://www.cs.bris.ac.uk/home/simonm/workshops/OpenCL_lecture3.pdf} 
\end{thebibliography}
}



\footnotesize{
\begin{thebibliography}{99}
 \bibitem[Label1, 2010]{key1}NVIDIA 
 \newblock OpenCL SDK Code Samples
 \newblock \url{https://developer.nvidia.com/opencl} 
\end{thebibliography}
}


\footnotesize{
\begin{thebibliography}{99}
 \bibitem[Label1, 2010]{key1} Vasily Volkov
(UC Berkeley , September 22, 2010)
 \newblock Better Performance at Lower Occupancy 
 \newblock \url{http://www.cs.berkeley.edu/~volkov/volkov10-GTC.pdf} 
\end{thebibliography}
}





\end{frame}
 
 
 
\begin{frame}
\centerline{Vielen Dank f{\"u}r eure Aufmerksamkeit!}
\end{frame}
% End of slides
\end{document}
