\documentclass{beamer}
\usepackage{etex}
\usepackage[UTF8]{inputenc}

\usepackage{beamerthemelined} 
\usepackage{amsmath}
\usepackage{mathtools}
\usepackage{listings}
\usepackage{graphicx}
\usepackage{epstopdf}
\usepackage{amsmath,amsfonts,amssymb}
\usepackage{upgreek}





\usepackage{algorithm}% http://ctan.org/pkg/algorithms
\usepackage{algpseudocode}% http://ctan.org/pkg/algorithmicx

\newcommand{\Input}{\item[\algorithmicinput]}
\newcommand{\algorithmicinput}{\textbf{Input:}}

\newcommand{\Output}{\item[\algorithmicoutput]}
\newcommand{\algorithmicoutput}{\textbf{Output:}}



\usepackage{tikz,pgfplots}
\usetikzlibrary{decorations.markings}
% \usepackage{courier}
% \usepackage[default,osfigures,scale=0.95]{opensans}

\usepackage{subcaption}
\usepackage{caption}
\usetikzlibrary{backgrounds}

\usepackage{subcaption}
\usepackage{caption}
\usetikzlibrary{backgrounds}


\setbeamertemplate{footline}[frame number]

% Syntax für C
\lstdefinestyle{customc}{
  belowcaptionskip=1\baselineskip,
  breaklines=true,
  frame=L,
  xleftmargin=\parindent,
  language=C++,
  showstringspaces=false,
  basicstyle=\footnotesize\ttfamily,
  keywordstyle=\bfseries\color{green!40!black},
  commentstyle=\itshape\color{purple!40!black},
  identifierstyle=\color{blue},
  stringstyle=\color{orange},
  tabsize=2,
}

%\usepackage{ stmaryrd }

\usetheme{Madrid}
%\usetheme{Boadilla} % Pretty neat, soft color.
%\usetheme{default}
%\usetheme{Warsaw}
%\usetheme{Bergen} % This template has nagivation on the left
%\usetheme{Frankfurt} % Similar to the default 
%with an extra region at the top.
%\usecolortheme{seahorse} % Simple and clean template
%\usetheme{Darmstadt} % not so good
% Uncomment the following line if you want %
% page numbers and using Warsaw theme%
 %\setbeamertemplate{footline}[page number]
%\setbeamercovered{transparent}
\setbeamercovered{invisible}
% To remove the navigation symbols from 
% the bottom of slides%
\setbeamertemplate{navigation symbols}{} 
%

\usepackage{graphicx}
%\usepackage{bm}         % For typesetting bold path (not \mathbold)
%\logo{\includegraphics[height=0.6cm]{yourlogo.eps}}
%
\title[Gröbner-Fächer für lineare Codes]{Gröbner-Fächer für lineare Codes}
\author{Daniel Rembold}
\institute[TUHH]
{
Technische Universit{\"a}t Hamburg Harburg \\
\medskip
{\emph{daniel.rembold@tuhh.de}}
}
\date{\today}
% \today will show current date. 
% Alternatively, you can specify a date.
%


\graphicspath{%
    {converted_graphics/}% inserted by PCTeX
    {/}% inserted by PCTeX
}
\begin{document}
% \renewcommand{\name}{Abbildung}
%
\begin{frame}
\titlepage
\end{frame}
%

\begin{frame}
\frametitle{Inhaltsverzeichnis}

\begin{enumerate}
			\item Einleitung \\

	      	\item Mathematische Grundlagen \\
	      	
	      	\item Aufzählen von Gröbner-Fächern  \\ 
	      		
	      	\item Ergebnisse  \\ 
	  
	      	\item Mögliche Verbesserungen \& Fazit \\

		    \item Vorführung
\end{enumerate}

\end{frame}

%%%%%%%%%%%%%%%%%%%%%%%%%%%%%%%%%% BEGIN INTRODUCTION %%%%%%%%%%%%%%%%%%%%%%%%%%%%%%%%%
\section{Einleitung}

% % % % % % % % % % % % % % % % % % % % % % % % % % % %

% % % % % % % % % % % % % % % % % % % % % % % % % % % %

\begin{frame}[<+->]
\frametitle{Motivation}

\begin{itemize}
\item
Viele Anwendungen der Gröbnerbasen basieren auf Gröbner-Fächer

\item
Nützlich für Idealzugehörigkeitsproblem und polynomiale Gleichungssysteme

\item 
Kompletter Gröbner-Fächer nicht immer nötig


\end{itemize}






\end{frame}

%%%%%%%%%%%%%%%%%%%%%%%%%%%%%%%%%%%%%%%%%%%%%%%%%%%%

\section{Mathematische Grundlagen}



%%%%%%%%%%%%%%%%%%%%%%%%%%%%
\subsection{Monome}

\begin{frame}[fragile]
\frametitle{Monome }

\begin{block}{Monom}
\begin{itemize}
\item Produkt von Variablen über einen endlichen Körper $ \mathbb{K} \left[X_{1},X_{2},\dots, X_{n}\right]  $ 
\item Schreibweise $m= X_{1}^{u_{1}}X_{2}^{u_{2}}\cdots X_{n}^{u_{n}}$ und $u_i \in \mathbb{N}_{0}  $
\end{itemize}

\end{block}
~\\
~\\
Grad eines Monoms:  $deg(m) = \sum_{i=1}^n u_i $. 




\end{frame}

% % % % % % % % % % % % % % % % % % % % % % % % % % % % %
\subsection{Termordnung}

\begin{frame}[<+->][fragile]
\frametitle{Termordnung (1)}
% --  Kommentar 
% --  Zu erst wird die Definition von Termordnung erklärt , dann werden die verschieden arten
\begin{block}{Termordnung $>$}
\begin{itemize}
\item Relation $>$ auf der Menge von allen Monomen in $ \mathbb{K} \left[X_{1},X_{2},\dots, X_{n}\right]  $ 
\end{itemize}

\end{block}

~\\

\begin{block}{Termordnung Beispiele }
\begin{itemize}
\item Lexikographische Ordnung $>_{lex}$
\item grad $>_{grlex}$
\item Ordnung mit Gewichtsvektor $c = (c_1, \ldots , c_n ) \in \mathbb{R}^{n}_{+}$
\end{itemize}



\end{block}


\end{frame}
% % % % % % % % % % % % % % % % % % % % % % % % % % % % % 
\subsection{Leitterm}


\begin{frame}[fragile]
\frametitle{Leitterm}

\begin{block}{Leitterm \textsc{LT}$(f)$ }
\begin{itemize}
\item Polynom $p \in  \mathbb{K} \left[X_{1},X_{2},\dots, X_{n}\right] $  besitzt Term höchster Ordnung in Bezug auf $>$
\end{itemize}

\end{block}


\begin{block}{Beispiel}
Sei $f = x^{2}+3xyz+y^{3} $
\begin{itemize}

\item lex-Order : $f = \underline{x^{2}} + 3xyz + y^{3}  $
\item grlex-Order :  $f = \underline{3xyz} + y^{3} + x^{2}  $
\item $\left(1,2,1\right)$ :  $f = \underline{y^{3}} + 3xyz + x^{2}  $
\end{itemize}

\end{block}


\end{frame}
% % % % % % % % % % % % % % % % % % % % % % % % % % % % %
\subsection{Ideale}

\begin{frame}[fragile]
\frametitle{Ideale}

\begin{block}{Ideal }
\begin{itemize}
\item Kollektion von Polynomen $f_{1}, \dots , f_{s}$ : \\
\[ \langle f_{1}, \dots , f_{s} \rangle = \left\lbrace  \sum_{i=1}^s h_{i}f_{i} \mid h_{1}, \dots , h_{s} \in \mathbb{K}\left[X_{1}, \dots, X_{n}\right] \right\rbrace. \] 
 
\end{itemize}

\end{block}

\begin{block}{Beispiel}
Sei $ I= \langle f_{1},f_{2} \rangle = \langle x^{2}+y, x+y+1 \rangle $ und $f=x^{2}y+x^{2}+y^{2}+xy+x$ \\
Es gilt $f= y \cdot f_{1} + x \cdot f_{2},~f\in I$.

\end{block}


\end{frame}
% % % % % % % % % % % % % % % % % % % % % % % % % % % % % % % % % % 
\subsection{Divisionsalgorithmus}
\begin{frame}
\frametitle{Divisionsalgorithmus (1)}



\begin{itemize}
\item Notwendig zum Lösen des Idealzugehörigkeitsproblems


\end{itemize}

\begin{block}{Ziel des Algorithmus}
Polynom g durch Ideal $ I = \langle f_{1}, \dots , f_{s} \rangle  $ teilen, so dass 
\[ g = a_{1}f_{1} + \ldots + a_{s}f_{s} + r ,~~~ a_{i},g,I,r \in \mathbb{K}\left[X_{1}, \dots, X_{n}\right]  \]
\end{block}

~\\

Sei Polynom $p$ und Ideal $I$  

\begin{itemize}
\item Wenn p \% I = 0, dann gilt $ p \in I$


\end{itemize}



	
\end{frame}
%%%%%%%%%%%%%%%%%%%%%%%%%%%


\begin{frame}[fragile]
\frametitle{Divisionsalgorithmus (2)}




\begin{algorithm}[H]
\caption{Divisionsalgorithmus  }

\begin{algorithmic}[1]

\Require Basis $I = \langle f_{1}, \dots, f_{m}\rangle$ von Polynomen $\neq$ 0 \textbf{und} ~~~~~~~~~~~~~~~~~~~~~~~~~~~~~~~~~ eine feste Termordnung \textsc{LT}  
\Ensure $r=0$ \textbf{or} none of the terms in $r$ are divisible by $ LT_{\leq}\left( f_{1}\right) , \dots , LT_{\leq} \left( f_{m}\right) $

%\EndProcedure
\end{algorithmic}
\end{algorithm}


% Hier Beispiel mit verdrehten Idealen zeigen
\begin{itemize}
\item Reihenfolge der Polynome in $I$ beeinflusst Ergebnis
\item $r \neq 0$ möglich, obwohl $p \in I$ 
\end{itemize}


\end{frame}



%  % %  % % % % % % % % % %  % % %
\subsection{Gröbnerbasis}


\begin{frame}[fragile]
\frametitle{Gröbnerbasis (1)}
%Kommentar : B


\begin{block}{Gröbnerbasis  }
Sei Termordnung $>$ und Ideal $I$, dann hat Gröbnerbasis $G = \{f_{1}, \ldots, f_{m} \}$ (in Bezug auf $>$) von $I$ die Eigenschaft: 
\begin{itemize}
\item Von jedem Polynom $p \in I$ ist $\textsc{LT}_{>}(p)$ teilbar durch $\textsc{LT}_{>}(f_{i})$  
\end{itemize} 
\end{block}

~\\
\begin{itemize}
\item Divisionsrest eindeutig bestimmt und unabhängig von Reihenfolge
~\\
\item Gröbnerbasis aus jedem Ideal mit Hilfe des Buchberger-Algorithmus und einer festen Termordnung 
\end{itemize}



\end{frame}



% % % % % % % % % % % % % % % % % % % % %
\subsection{Gröbnerbasis}


\begin{frame}[fragile]
\frametitle{Gröbnerbasis (2)}
%Kommentar 

\begin{itemize}
\item Gröbnerbasen sind nicht eindeutig

\item Reduzierte Gröbnerbasen jedoch eindeutig für Ideale
\end{itemize}

\begin{block}{Reduzierte Gröbnerbasis}
Alle Leitterme von Gröbnerbasis G in Bezug auf Termordnung $>$ monisch und relativ prim zueinander 
\end{block}




\end{frame}







% % % % % % % % % % % % % % % % % % % % % % % % % %
\subsection{Gröbner-Fächer}


\begin{frame}[fragile]
\frametitle{Gröbner-Fächer}
%Kommentar: Text der Bachelorarbeit
\begin{itemize}
\item Unendlich viele Termordnungen, endlich viele reduzierte Gröbnerbasen

\end{itemize}

\begin{block}{Gröbner-Fächer}
\begin{itemize}
\item Vielflächiges Komplex welches Kegel im $\mathbb{R}^{n}_{+}$ enthält
\item Kegel werden durch lineare Ungleichungen der Polynome bestimmt 
\end{itemize}

\end{block}

\end{frame}
% % % % % % % % % % % % % % % % % % % % % % % % % %

\begin{frame}[fragile]
\frametitle{Beispiel Gröbner-Fächer (1)}
 Sei 
 \begin{enumerate}
 \item $I = \langle x^{2}-z,~y-x  \rangle \in \mathbb{K}\left[x,y,z\right]  $
 \item $G_{>_{lex}} = \{ \underline{y^{2}}-z,~\underline{x}-y  \} $ 
 \item $\textbf{w}= (a,b,c) \in \mathbb{R}^{3}_{+} $
\end{enumerate}
 
~\\
\begin{itemize}
\item $w \in G_{>_{lex}}$ genau dann, wenn 
~\\
~\\
\begin{itemize}
\item $ \left(0,2,0\right) \cdot \left(a,b,c\right) \geq \left(0,0,1\right) \cdot  \left(a,b,c\right) ~ \vee ~ 2b \geq c  $
~\\
~\\
\item
$ \left(1,0,0\right) \cdot \left(a,b,c\right) \geq \left(0,1,0\right) \cdot  \left(a,b,c\right) ~ \vee ~ a \geq b  $
\end{itemize}
\end{itemize}

\end{frame}


% % % % % % % % % % % % % % % % % % % % % % % % % %
\begin{frame}[fragile]
\frametitle{Beispiel Gröbner-Fächer (2)}

\begin{center}


\begin{figure}
\caption{Gröbner-Kegel für $G_{>_{lex}}$}

\begin{tikzpicture}[join=round,scale=0.8]
    \tikzstyle{conefill} = [fill=blue!20,fill opacity=0.8]
    \tikzstyle{ann} = [fill=white,font=\footnotesize,inner sep=1pt]
    \tikzstyle{ghostfill} = [fill=white]
    \tikzstyle{ghostdraw} = [draw=black!50]
    
    \draw[arrows=->,line width=.4pt](0,0,0)--(0,0,5); %Z_achse
    \draw[arrows=->,line width=.4pt](0,0,0)--(0,5,0); %Y-ACHSE
    \draw[arrows=->,line width=.4pt](0,0,0)--(5,0,0); %X-ACHSE
    %\draw[arrows=<-,line width=.4pt](.42,-.767)--(4,-2);
    
    \path (5,0,0) node[below] {$b$} (0,0,5) node[above] {$a$} (0,5,0) node[left] {$c$};
    
% letzte Koordinate ist A!!!
% zweite Koordinate ist C!!!
% dritte Koordinate ist B!!!
  
\filldraw[conefill](0,0,0)--(0,0,4)--(1,2,1)--cycle;
\filldraw[conefill](1,2,1)--(0,0,4)--(2,0,2)--cycle;
%\filldraw[conefill](1,2,1)--(0,0,0)--(2,0,2)--cycle;

\draw [opacity=0.2] (0,0,0) -- (2,0,2) ;
 
   
\end{tikzpicture}

\end{figure}

\end{center}



\end{frame}

% % % % % % % % % % % % % % % % % % % % % % % % % % 
\begin{frame}[fragile]
\frametitle{Beispiel Gröbner-Fächer (3)}

\begin{center}


\begin{figure}
\caption{Kompletter Gröbner-Fächer}

\begin{tikzpicture}[join=round,scale=0.8]
    \tikzstyle{conefill} = [fill=blue!20,fill opacity=0.8]
    \tikzstyle{ann} = [fill=white,font=\footnotesize,inner sep=1pt]
    \tikzstyle{ghostfill} = [fill=white]
    \tikzstyle{ghostdraw} = [draw=black!50]
    
    \draw[arrows=->,line width=.4pt](0,0,0)--(0,0,5); %Z_achse
    \draw[arrows=->,line width=.4pt](0,0,0)--(0,5,0); %Y-ACHSE
    \draw[arrows=->,line width=.4pt](0,0,0)--(5,0,0); %X-ACHSE
    %\draw[arrows=<-,line width=.4pt](.42,-.767)--(4,-2);
    
    \path (5,0,0) node[below] {$b$} (0,0,5) node[above] {$a$} (0,5,0) node[left] {$c$};
   
% erste Koordinate ist B!!! 
% zweite Koordinate ist C!!!   
% dritte Koordinate ist A!!!


% äußere Flächen!

\filldraw[conefill](0,0,0)--(0,4,0)--(0,0,4)--cycle;
\filldraw[conefill](0,0,0)--(4,0,0)--(0,0,4)--cycle;
\filldraw[conefill](0,0,0)--(4,0,0)--(0,4,0)--cycle;

% untenfläche
\filldraw[conefill](0,0,0)--(0,0,4)--(2,0,2)--cycle;
\filldraw[conefill](0,0,0)--(4,0,0)--(2,0,2)--cycle;

%
\filldraw[conefill](0,0,4)--(2,0,2)--(1,2,1)--cycle;
\filldraw[conefill](4,0,0)--(2,0,2)--(1,2,1)--cycle;

\filldraw[conefill](4,0,0)--(1,2,1)--(0,4,0)--cycle;
\filldraw[conefill](0,0,4)--(1,2,1)--(0,4,0)--cycle;

% zur Übersichtlichkeit die gerade im "inneren" des Cones
\draw [opacity=0.2] (0,0,0) -- (1,2,1) ;
\end{tikzpicture}


\end{figure}

\end{center}


\end{frame}



% % % % % % % % % % % % % % % % % % % % % % % % % %
\subsection{Torische Ideale}

\begin{frame}[fragile]
\frametitle{Torische Ideale}

\begin{itemize}
\item Binomiales Ideal
\end{itemize}
~\\
\begin{block}{Torisches Ideal}
Gegeben seien $A =\left[a_{1},\dots, a_{n}  \right] \in \mathbb{Z}^{d \times n } $ und $u \in \mathbb{Z}^{n}$ zerlegbar in $u^{+}$ und $u^{-}$, dann ist das torische Ideal $I_{A}$ definiert durch \\
~\\
\centering
$ \textbf{I}_{A} = \langle \textbf{x}^{u^{+}} - \textbf{x}^{u^{-}} \mid u \in ker \left(  A \right) \rangle . $ 

\end{block}


\end{frame}
% -------------------------------------------------------- % % %
\section{Aufzählen von Gröbner-Fächern}

% % % % % % % % % % % % % % % % % % % % % % % % % % % % % % % % %
\subsection{Facet Binomials}
\begin{frame}
\frametitle{Facet Binomials }

\begin{itemize}
\item Gröbner-Fächer als Graph
~\\
\begin{itemize}
\item Gröbnerbasen als Knoten
\item Gröbnerbasen adjazent wenn im Gröbner-Fächer benachbart
\end{itemize}
\end{itemize}

\begin{block}{Facet Binomials}
Sei $x^{\upalpha_{k}}-x^{\upbeta_k} \in \mathcal{G}_c~$, $\mathcal{G}_c$ Gröbnerbasis mit Gewichtsvektor $c$\\
$x^{\upalpha_{k}}-x^{\upbeta_k}$ ist Facet binomial wenn ein Vektor $u \in \mathbb{R}^{n}$ folgendes erfüllt : \\

\begin{align*}
   \upalpha_{i} \cdot u & \geq \upbeta_{i}  \cdot u : i = 1, \dots , t, i \neq k \\
   \upbeta_{k}  \cdot u & \geq \upalpha_{k}  \cdot u
\end{align*}


\end{block}

\end{frame}

\begin{frame}
\frametitle{Beispiel Facet Binomials }

\begin{itemize}
\item Sei $\mathcal{G} = \{x_1 - x_5,~x_{2}^{2}-1,~x_{2}x_{4}-x_{5}x_{6},~x_{2}x_{6}-x_{4}x_{5},~x_{3}-x_{5},~x_{4}^{2}-1,~x_{4}x_{5}x_{6}-x_{2},~x_{5}^{2}-1,~x_{6}^{2}-1  \}~$
\item Prüfe ob $x_{2}x_{4}-x_{5}x_{6}~$ ein facet binomial ist mit LP
\end{itemize}

 \[
 	\begin{array}{lrcl}
 	\textrm{ }   & Ax    & =    & b   \\
 	\textrm{so dass}  & x     & \geq & 0
 			    
 	\end{array}
 \]
 
\begin{enumerate}
\item  $b = {\left(0,1,0,1,-1,-1\right)}^{T}$
\item  \[
 A =
 \begin{pmatrix}
 1  & 0 & 0  & 0 & 0 & 0  & 0 & 0\\ 
 0  & 2 & 1  & 0 & 0 & -1 & 0 & 0\\  
 0  & 0 & 0  & 1 & 0 & 0  & 0 & 0\\ 
 0  & 0 & -1 & 0 & 2 & 1  & 0 & 0\\
 -1 & 0 & -1 & 1 & 0 & 1  & 2 & 0\\
 0  & 0 & 1  & 0 & 0 & 1  & 0 & 2
 \end{pmatrix} 
 \] 
\end{enumerate}


\end{frame}

% % % % % % % % % % % % % % % % % % % % % % % % % % % % % % % %
\subsection{Traversieren von Gröbnerbasen}



\begin{frame}
\frametitle{Traversieren von Gröbnerbasen}
\begin{itemize}
\item Gröberbasis erhalten durch andere Gröbnerbasis
\item Facet Binomial "umdrehen" (flip)
\end{itemize}


\begin{algorithm}[H]
\caption{Local change of reduced Gröbner bases in $I_A$ }
\label{alg:flip}
\begin{algorithmic}[1]

\Input
Reduced Gröbner basis $ \mathcal{G} $ of $I_A$ \textbf{and}

    A prescribed facet binomial $ \underline{x}^{a}_{i} - x^{b}_{i} \in \mathcal{G} $
%\Comment {The weight vector inducing $\mathcal{G}$ is generic and the leading terms are underlined}
\Output The reduced Gröbner basis adjacent to $\mathcal{G}$ in which $ \underline{x}^{b}_{i} - x^{a}_{i} $ is a facet binomial.


%\EndProcedure
\end{algorithmic}
\end{algorithm}

\begin{itemize}
\item Keine expliziten Termordnung nötig

\end{itemize}

	
\end{frame}

% % % % % % % % % % % % % % % % % % % % % % % % % % % % %

\begin{frame}[fragile]
\frametitle{Breitensuche}


\begin{algorithm}[H]
\caption{Enumerating the edge graph of the Gröbner fan via breath-first search}
\label{alg:breath}
\begin{algorithmic}[1]

\Input
Any reduced Gröbner basis $ \mathcal{G}_0 $ of $I_A$
\Output All reduced Gröbner bases of $I_A$, (all vertices of the edge graph)

%\EndProcedure
\end{algorithmic}
\end{algorithm}

\begin{itemize}
\item Einfacher und intuitiver Algorithmus
\item Alle Gröbnerbasen müssen für den Vergleich gespeichert werden
\end{itemize}
 
\end{frame}





% % % % % % % % % % % % % % % % % % % % % % % % %

\begin{frame}[fragile]
\frametitle{Tiefensuche(1)}

\begin{itemize}
\item Gröbner-Fächer mit umgekehrter Tiefensuche nummerieren
\item Ergebnis ist ein gerichter Subgraph $\longrightarrow$ Umgekehrter Suchbaum 
\item Nachteil von Breitensuche kompensiert

\end{itemize}

\begin{block}{Umgekehrter Suchbaum $T_{>}(I_{A})$}
\begin{itemize}
\item Azyklischer Graph mit einer eindeutigen Senke (bzgl. einer festen Termordnung)
\item Eindeutige Pfade von Gröbnerbasis zur Senke
\end{itemize}

\end{block}
 
\end{frame}

% % % % % % % % % % % % % % % % % % % % % % % % %


\begin{frame}[fragile]
\frametitle{Tiefensuche(2)}


\begin{algorithm}[H]
\caption{Enumerating the edge graph of the Gröbner fan via reverse search}
%Kommentar erklären
% Anhand der facet-binomials und der Termordnung werden die facets markiert die geflippt werden sollen
% Anschließend geflippt bis alle kein facet mehr übrig ist
% dann wird der Baum zuwückverfolgt und falls man bis zur Senke zurückgehen musste UND keine facets mehr übrig sind zum flippen, dann terminiert algorithmus

\begin{algorithmic}[1]

\Input
Any reduced Gröbner basis $ \mathcal{R}_{>} $ of $I_A$ and its term order $>$
\Output All reduced Gröbner bases of $I_A$ (all vertices of the edge graph)


%\EndProcedure
\end{algorithmic}
\end{algorithm}

\begin{itemize}
\item Termordnung notwendig für Tiefensuche
\item Binome die Termordung einhalten werden "umgedreht"
\end{itemize}

 
\end{frame}

% % % % % % % % % % % % % % % % % % % % % % % % %

\begin{frame}
\frametitle{Vergleich: Breitensuche \& umgekehrte Tiefensuche}

Sei $ \mathcal{G} = \{x_{1} - x_{2}, x_{3} - x_{4}, x_{5}-x_{6} , x_{2}^{2} -1 , x_{4}^{2} - 1, x_{6}^{2} - 1 \} $ mit lexikographischer Ordnung $>$ : $x_{1} > \ldots > x_{6} $


\begin{figure}[h]
    \centering
    \begin{subfigure}[b]{0.48\linewidth}        %% or \columnwidth
        \centering
        \begin{tikzpicture}[scale = 0.5]
\node (A) at (0,0) {$\mathcal{G}_{1}$};

\node (C) at (-3,-3) {$\mathcal{G}_{2}$};
\node (D) at (0,-3) {$\mathcal{G}_{4}$};
\node (B) at (3,-3) {$\mathcal{G}_{8}$};

\node (E) at (3,-6) {$\mathcal{G}_{7}$};
\node (F) at (-3,-6) {$\mathcal{G}_{3}$};
\node (G) at (0,-6) {$\mathcal{G}_{5}$};

\node (H) at (0,-9) {$\mathcal{G}_{6}$};


\draw[-, very thick] (D) to (A);
\draw[-, very thick] (C) to (A);
\draw[-, very thick] (B) to (A);

\draw[-, very thick] (E) to (D);
\draw[-, very thick] (F) to (C);
\draw[-, very thick] (G) to (D);
\draw[-, very thick] (G) to (C);
\draw[-, very thick] (E) to (D);
\draw[-, very thick] (F) to (B);
\draw[-, very thick] (E) to (B);

\draw[-, very thick] (H) to (G);
\draw[-, very thick] (H) to (F);
\draw[-, very thick] (H) to (E);
%\draw[-, very thick] (F) to (G);

\end{tikzpicture}
        \caption{Ergebnis der Breitensuche}
        \label{fig:breadth}
    \end{subfigure}
    \begin{subfigure}[b]{0.48\linewidth}        %% or \columnwidth
        \centering
        
\begin{tikzpicture}

\node (A) at (0,0) {$\mathcal{G}_{1}$};

\node (C) at (-2,-3) {$\mathcal{G}_{2}$};
\node (D) at (0,-3) {$\mathcal{G}_{4}$};
\node (B) at (2,-3) {$\mathcal{G}_{8}$};

\node (E) at (2,-6) {$\mathcal{G}_{7}$};
\node (F) at (-2,-6) {$\mathcal{G}_{3}$};
\node (G) at (0,-6) {$\mathcal{G}_{5}$};

\node (H) at (0,-9) {$\mathcal{G}_{6}$};


\draw[->, very thick] (D) to node[midway, above, sloped]  {$x_{6}-x_{5}$}(A);
\draw[->, very thick] (C) to node[midway, above, sloped]   {$x_{4}-x_{3}$} (A);
\draw[->, very thick] (B) to node[midway, above, sloped]   {$x_{2}-x_{1}$} (A);

\draw[->, very thick] (E) to node[midway, above, sloped]   {$x_{2}-x_{1}$} (D);
\draw[->, very thick] (F) to node[midway, above, sloped]   {$x_{2}-x_{1}$}(C);
\draw[->, very thick] (G) to node[midway, above, sloped]   {$x_{4}-x_{3}$}(D);

\draw[->, very thick] (H) to node[midway, above, sloped]   {$x_{2}-x_{1}$}(G);


\end{tikzpicture}

        \caption{Umgekehrter Suchbaum}
        \label{fig:reverse}
    \end{subfigure}
    %\caption{Vergleich zwischen the breath-first search and the reverse search}
    \label{fig:graph}
\end{figure}

\end{frame}

% % % % % % % % % % % % % % % % % % % % % % % % %
\subsection{Gradkompatible Gröbnerbasis}
\begin{frame}
\frametitle{Gradkompatible Gröbnerbasis (1)}

\begin{block}{Gradkompatible Gröbnerbasis}
Eine reduzierte Gröbnerbasis (bzgl. zur Termordnung $>$) für ein Ideal ist gradkompatibel wenn der Vektor \textbf{1} im Gröbner-Kegel liegt.

\end{block}

\begin{itemize}
\item Leitterm muss höchsten Grad haben
\item Jedes Ideal hat mindestens eine gradkompatible Gröbnerbasis
\end{itemize}




\end{frame}

% % % % % % % % % % % % % % % % % % % % % % % % %

\begin{frame}
\frametitle{Gradkompatible Gröbnerbasis (2)}


\begin{block}{Einzige gradkompatible Gröbnerbasis}
Eine reduzierte Gröbnerbasis (bzgl. einer Termordnung $>$) ist die einzige gradkompatible Gröbnerbasis wenn 
\[ deg(x^{a}) > deg(x^{b})~ \forall~ x^{a}-x^{b}\in \mathcal{G} \] 

\end{block}
~\\
\begin{itemize}
\item Breitensuche und umgekehrte Tiefensuche adaptierbar auf gradkompatible Gröbnerbasen
\end{itemize}

\end{frame}


% % % % % % % % % % % % % % % % % % % % % % % % % 
\subsection{Lineare Codes}

\begin{frame}
\frametitle{Lineare Codes}


\begin{block}{Linearer Code}
Ein linearer Code $\mathcal{C}$  der Länge n und Dimension k über $\mathbb{F}$ ist eine injektive lineare Abbildung  $\phi~:~\mathbb{F}^{k} \rightarrow \mathbb{F}^{n},~n\geq k $.
\end{block}

\begin{itemize}
\item Als [$n,k$] Code bezeichnet
\item Alternativ als Generatormatrix $G \in \mathbb{F}^{k \times n}$ beschrieben 
~\\
\item Standardform : $G = (I_{k}| M)$ % I_k kxk Einheitsmatrix
\item Codewort $c~$vom Wort $x~$erhält man durch 
\[
     xG = c 
 \]
\end{itemize}

\end{frame}

% % % % % % % % % % % % % % % % % % % % % % % % % 
\begin{frame}[fragile]
\frametitle{Beispiel lineare Codes}

\begin{itemize}
\item   Sei $x = \left(1,0,1,0\right)$ und $
 G =
 \begin{pmatrix}
 1 & 0 & 0 & 0 & 1 & 1 & 0 \\ 
 0 & 1 & 0 & 0 & 1 & 0 & 1 \\  
 0 & 0 & 1 & 0 & 1 & 1 & 1 \\ 
 0 & 0 & 0 & 1 & 0 & 1 & 1
 \end{pmatrix} 
 $
\end{itemize}

 ~\\
 ~\\
 \[
      \left(1,0,1,0\right) \cdot \begin{pmatrix}
      1 & 0 & 0 & 0 & 1 & 1 & 0 \\ 
      0 & 1 & 0 & 0 & 1 & 0 & 1 \\  
      0 & 0 & 1 & 0 & 1 & 1 & 1 \\ 
      0 & 0 & 0 & 1 & 0 & 1 & 1
      \end{pmatrix}   = \left(1,0,1,0,0,0,1\right) 
  \]




\end{frame}


% % % % % % % % % % % % % % % % % % % % % % % % %
\frametitle{Code Ideal}
\begin{frame}
\frametitle{Code Ideal}

\begin{block}{Code Ideal}
Sei $\mathcal{C}$ ein $[n,k]$ Code. Das Code Ideal $I(\mathcal{C})$ die Vereinigung zwischen

\begin{align*}
 I_{\mathcal{C}} & = \langle \textbf{x}^{c} - \textbf{x}^{c'} | c - c' \in \mathcal{C}  \rangle + I_{p},\\
\textrm{wobei} ~ I_{p} & = \langle x_{i}^{p} - 1 | 1 \leq i \leq n \rangle .
\end{align*}


\end{block}



\end{frame}


% % % % % % % % % % % % % % % % % % % % % % % % %

\begin{frame}
\frametitle{Beispiel Code Ideal}
\begin{itemize}
\item 
Sei $\mathcal{C}$ ein binärer $[6,3]$ code mit der Generatormatrix
\[
G_{1} =
\begin{pmatrix}
1 & 0 & 0 & 0 & 1 & 0 \\ 
0 & 1 & 0 & 1 & 1 & 1 \\  
0 & 0 & 1 & 0 & 1 & 0  
\end{pmatrix} 
\]

\item Code Ideal $I(\mathcal{C})$ ergibt sich zu: 
\begin{align*}
I(\mathcal{C}) = &~\{x_{1}-x_{5},~x_{2}-x_{4}x_{5}x_{6},~x_{3}-x_{5}  \} ~\cup \\ &~\{x_{1}^{2}-1,~x_{2}^{2}-1,~x_{3}^{2}-1,~x_{4}^{2}-1,~x_{5}^{2}-1,~x_{6}^{2}-1\}
\end{align*}
\item Reduzierte Gröbnerbasis mit lexikographischer Ordnung $>$ :
\begin{center}
$ \mathcal{G}_{>} = \{x_{1}-x_{5},~x_{2}-x_{4}x_{5}x_{6},~x_{3}-x_{5}  \} \cup \{x_{4}^{2}-1,~x_{5}^{2}-1,~x_{6}^{2}-1  \}  $
\end{center} 

\end{itemize}
\end{frame}

% % % % % % % % % % % % % % % % % % % % % % % % %
\section{Ergebnisse}

\begin{frame}[fragile]
\frametitle{Vergleich zw. gradkompatible \& kompletter Gröbner-Fächer}

\begin{figure}[h]
\begin{tikzpicture}[scale = 0.7]
  \centering
  \begin{axis}[
  		ybar, axis on top,
       % title={Example Text},
        height=8cm, width=13.5cm,
        bar width=0.4cm,
        ymajorgrids, tick align=inside,
        major grid style={dashed,gray},
        enlarge y limits={value=.1,upper},
        ymin=0, ymax=2300,
        axis x line*=bottom,
        axis y line*=right,
        y axis line style={opacity=1},
        tickwidth=0pt,
        enlarge x limits=true,
        legend style={
            at={(0.5,-0.2)},
            anchor=north,
            legend columns=-1,
            /tikz/every even column/.append style={column sep=0.5cm}
        },
        xlabel={Dimension $k$ of the linear code},
        ylabel={Number of reduced Gröbner bases},
        symbolic x coords={ 1,2,3,4,5,6,7},
       xtick=data,
       nodes near coords={
        \pgfmathprintnumber[precision=0]{\pgfplotspointmeta}
       }
    ]
    \addplot [draw=none, fill=blue!30] coordinates {
      (1,8)
      (2,8) 
      (3,16)
      (4,162) 
      (5,12) 
      (6,18)
      (7,8)};
      
   \addplot [draw=none,fill=red!30] coordinates {
      (1,12)
      (2,1708) 
      (3,328)
      (4,2257) 
      (5,224) 
      (6,39)
      (7,8)};
      

    \legend{Degree compatible,All Gröbner bases}
  \end{axis}
  \end{tikzpicture}
 % \caption{Comparison between the numbers of degree compatible and all Gröbner bases of binary linear codes with the length 8 }
  \label{fig:numbergb}
  \end{figure}



\end{frame}


% % % % % % % % % % % % % % % % % % % % % % % % %
\begin{frame}
\frametitle{Zeitvergleich zwischen Gfan \& Software }

 \begin{table}[htpb] 
 \centering 
 \caption{Berechnungszeit in Sekunden}
 \begin{tabular}{c|l|l|l} % zentriert rest linksbündig
   $[n,k]$ Code & CIDGEL d.c. & CIDGEL & Gfan \\
 \cline{1-4} 
 			 $[8,2]$  & 0.206 & 10.198 & 38.127 \\
  			 $[8,4]$  & 7.743 & 25.86  & 47.748 \\
 			 $[9,4]$  & 9.27  & 727.91 & 982.56 \\
  			 $[9,5]$  & 4.72  & 18.89  & 59.65 \\
  			 $[10,6]$ & 87.92 & 277.81 & 380.04 \\
 \end{tabular}
 \label{tab:benchmark}
 \end{table}  


\end{frame}

% % % % % % % % % % % % % % % % % % % % % % % % %
\section{Vorführung}
\begin{frame}

\centerline{Vorführung der Software}


\end{frame}
% % % % % % % % % % % % % % % % % % % % % % % % %

\section{Mögliche Verbesserungen \& Fazit}
 
\begin{frame}
\frametitle{Mögliche Verbesserungen}


\begin{enumerate}
\item Verwenden von externen LP-Solver %Hinweis darauf das facetberechnung  sehr teuer sind
~\\ ~\\
\item Parallelität
\end{enumerate}


\end{frame}


\begin{frame}
\frametitle{Fazit}
\begin{enumerate}
\item Keine (bekannte) Software berechnet gradkompatiblen Gröbnerfächer %Hinweis darauf das facetberechnung  sehr teuer sind
~\\ ~\\
\item Schnellere Berechnung als andere Software
\end{enumerate}


\end{frame} 
 
 
% % % % % % % % % % % % % % % % % % % % % % % % %
 
 
\begin{frame}
\centerline{Vielen Dank f{\"u}r eure Aufmerksamkeit!}
\end{frame}
% End of slides
\end{document}
